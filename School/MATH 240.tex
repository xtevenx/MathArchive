\documentclass{article}


\usepackage{mdframed}
\newmdtheoremenv{theorem}{Theorem}

%\usepackage{xcolor}
%\pagecolor[HTML]{212121}
%\color[HTML]{FFFFFF}
%
%\definecolor{foreground}{HTML}{FFFFFF}
%\definecolor{background}{HTML}{212121}
%\newmdtheoremenv[
%    fontcolor=foreground, backgroundcolor=background,
%    linecolor=foreground
%    ]{theorem}{Theorem}


\begin{document}


\section{Systems of Linear Equations}


\subsection{Definitions}

\begin{itemize}

\item
A \textbf{linear equation} in variables $ x_1, \dots, x_n $ is of the form
$ a_1 x_1 + x_2 x_2 + \dots a_n x_n = b $ where $ a_1, \dots, a_n, b $ are
numbers.

\item
A \textbf{system} of linear equations in variables $ x_1, \dots, x_n $ is a
set of one or more linear equations.

\item
A \textbf{solution} to a system of linear equations is a set of numbers such
that when plugged in for the variables, the equations are satisfited.

\item
Two systems of linear equations sharing the same solution(s) are considered to
be \textbf{equivalent}.

\item
A system with any nonzero number of solutions is \textbf{consistent}.

\item
A system of linear equations with no solutions is \textbf{inconsistent}.

\item
A \textbf{coefficient matrix} is a matrix holding the coefficients of a system
of linear equations.

\item
An \textbf{augmented matrix} is a coefficient matrix with an additional column
on the right holding the constants of each linear equation.

\end{itemize}


\subsection{Lecture Material}

There are three basic operations which can be done on a system of linear
equations called \textbf{elementary row operations}:

\begin{enumerate}

\item
\textbf{Interchange} the order of two rows in the system.

\item
\textbf{Scale} all terms in a row by the same nonzero constant.

\item
\textbf{Replace} a row by adding a multiple of another row to it.

\end{enumerate}


\section{Row Reduction and Echelon Forms}


\subsection{Definitions}

\begin{itemize}

\item
The \textbf{leading entry} of a row is the first nonzero term in that linear
equation.

\item
A matrix is in \textbf{echelon form} if it has the following properties:

\begin{enumerate}

\item
Any rows of all zero are at the bottom.

\item
The leading entry of each row is to the right of the leading entry to each row
above it.

\end{enumerate}

\item
A matrix is in \textbf{reduced echelon form} if it is in echelon form and has
the following additional property:

\begin{enumerate}

\item
The leading entry in each nonzero row is 1 and also the only nonzero
entry in that column.

\end{enumerate}

\item
A \textbf{pivot position} in matrix $ M $ is a position in $ M $ that holds a
leading 1 of the corresponding reduced echelon matrix of $ M $.

\item
A \textbf{pivot column} is a column that contains a pivot position

\item
A \textbf{basic variable} is a variable in a pivot column.

\item
A \textbf{free variable} is a variable not in a pivot column. This means that
the variable can be set to any value and still be part of a solution.

\item
A \textbf{parametric description} of a solution set is one in which the basic
variables are described in terms of the free variables and the free variables
are listed as such. An empty solution set has no parametric description.

\end{itemize}


\subsection{Lecture Material}

\begin{theorem}
There is only one reduced echelon matrix corresponding to each matrix.
\end{theorem}

The \textbf{row reduction algorithm} is a five-step plan to acing your linear
algebra course:\footnote{If you don't think about this too much it'll make you feel better.}

\begin{enumerate}

\item
The leftmost nonzero column is a pivot column with the pivot position in the
first row.

\item
Interchange rows such that a non-zero entry is in the pivot position.

\item
Use replacement operations to set all entries below the pivot to 0.

\item
Ignore everything at or above the row that was just operated on. If the matrix
is not in echelon form, go back to step one.

\item
Starting from the bottommost row, use replacement operations to set all entries
above the pivot to 0. Repeat this until the matrix is in reduced echelon form.

\end{enumerate}

\begin{theorem}
A linear system is consistent if and only if none of the equations (in echelon
form) describe zero to be equal to a nonzero constant. A consistent linear
system contains one unique solution if and only if there are no free variables,
otherwise, it has an infinite number of solutions.
\end{theorem}

The following steps can be done to find all the solutions of a linear system
with row reduction:

\begin{enumerate}

\item
Use row reduction to obtain an equivalent matrix in echelon form. If there is
no solution (this can be determined with Theorem 2), stop here, do not pass go,
and do not collet \$200.

\item
Continue row reduction to obtain the reduced echelon form.

\item
Write the system of equations that corresponding to the reduced echelon matrix.

\item
Solve each basic variable in terms of the free variables.

\end{enumerate}


\end{document}

