\documentclass{article}
\usepackage{amsmath}


\title{Sum of Finite Geometric Series}
\author{Steven Xia}


\begin{document}
\maketitle

\section{Statement}
$$\sum_{k=0}^n b^k = \frac{b^{n+1} - 1}{b - 1}$$

\section{Proof}

\subsection{Verification by Intuition}
Written in base $b$, $b^k$ is represented as a one followed by $k$ zeroes.
As such, the sum in the statement above is represented as $n+1$ ones in base $b$.

Continuing, $b^{n+1}$ is represented as a one followed by $n+2$ zeroes.
This means that $b^{n+1} - 1$ is represented as $n+1$ characters representing the value of $b - 1$ (in the case of base ten, this character would be nine).
Finally, dividing the number by $b-1$ results in $n+1$ ones, which is equal to the value of the sum as shown initially.

\subsection{Proof by Induction}

The base case, $n=0$, is trivial; both sides evaluate to 1.

\subsubsection{Inductive step: $n\geq1$}
\begin{align}
    \frac{b^{n+1} - 1}{b - 1} &= \frac{b^n - 1}{b - 1} + b^n \\
                              &= \frac{b^n - 1 + (b - 1)b^n}{b - 1} \\
                              &= \frac{(1 + (b - 1))b^n - 1}{b - 1} \\
                              &= \frac{(b)b^n - 1}{b - 1} \\
                              &= \frac{b^{n+1} - 1}{b - 1}
\end{align}

\end{document}
